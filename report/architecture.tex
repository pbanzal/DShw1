\section{Architecture}
\label{sec:arch}

A reliable communications channel is duplex by nature. At the
high-level, the channel is composed of two components - sender and
receiver. The sender component accepts a message, which is a string of
arbitrary length, converts the message to byte stream followed by
fragmentation into UDP datagrams before sending. The sender requires
acknowledgement (ACK) from receiever acknowledging the receipt of
every datagram. In order to match acknowldgements with datagrams,
every datagram is assigned a unique identifier. To facilitate FIFO
order at the receiver, unique identifiers are assigned sequentially,
hence they are sequence numbers.

Sender maintains an \emph{UNACK\_SET} of datagrams that are yet to be
acknowledged by the receiver. Considering that network bandwidth and
memory at receiver are both finite resources, the size of
\emph{UNACK\_SET} is bound. If the bound is reached, sender goes to
sleep. It periodically wakes up to see if any datagrams in
\emph{UNACK\_SET} are acknowledged. If there are any such datagrams, it
removes them from the set, resends the rest, and fills the gap by
sending fresh datagrams, if there are any. Sender repeats this till
there are no more datagrams to send.

The task of the receiver component of the channel is to receive and
reconstruct the message sent by sender. Since message is fragmented,
and fragments have sequence numbers, receiver maintains a
\emph{sliding window} of \emph{holes} denoting consecutive sequence numbers
of fragments that it is expecting. If a new datagram with sequence
number in the expected range (i.e., current window) arrives, it is
acknowledged and corresponding \emph{hole} is filled. If a datagram with sequence number less
than expected range of the current window arrives, it means the sender has resent a frame, either because it woke up earlier than ack reached it/ack was lost. In this case, we resend ack for that fragment. If the sequence
number is lowest of the expected range, the bounds of the range are
incremented by one (i.e., window slides to right). If the arrived
fragment is the last of the fragments of a message, a flag is set to
denote the same. Once such fragment arrives, given that all holes
corresponding to previous fragments are filled, the message is
reconstructed and handed over to the listener.
