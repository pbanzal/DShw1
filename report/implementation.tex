\section{Implementation}
\label{sec:impl}

\code{RChannel} class implements the interface \code{ReliableChannel}.
The constructor of the class accepts local port, destination port and
destination IP, and opens a UDP socket on the local port. Sender and
receiver components manifest as separate threads of execution
implemented by \code{SenderThread} and \code{ReceiverThread} classes
respectively. Since channel has to continue accepting messages from
the user it continues executing the main thread. To dispatch
user messages to sender thread, the channel thread uses a
\code{sendBuffer} linked list that is shared between both the threads.
Java construct \code{synchronize} was used to control the exclusive
access to the buffer. 

The size of payload in each message fragment is restricted to a fixed
number, which is 600 unicode characters, or 1200 bytes. Apart from the
payload, the message also contains its sequence number, and flags to
denote if it is an acknowledgement, or if it is contains the final
fragment of a message. The message is implemented by \code{RMessage}
class, which also implements \code{Serializable} class, so as to
enable conversion to byte stream. 

The \code{ReceiverThread} maintains expected sequence number through
\code{RChannel} class, which denotes the starting sequence number
(left-most hole) of the \emph{sliding window}. When a datagram with
expected sequence number arrives, the expected number is incremented
denoting the sliding of the window. \code{bufferLength}
variable denotes the size of the sliding window. Any datagram that
falls in the sliding window is acknowledged by sending an
\code{RMessage} with same sequence number as the datagram, but with
ACK flag set.

Since ACK itself is a message, the \code{ReceiverThread} on sender
side is also responsible to convey acknowledgements to sender thread.
It does so by marking corresponding messages in \code{sendBuffer} with
ACK flag. Therefore, \code{sendBuffer} is also shared with
\code{ReceiverThread}.

Once the message is reconstructed by \code{ReceiverThread}, it calls
the callback to convey the message to the listener.
